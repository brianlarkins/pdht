\vspace{-1mm}

\section{Conclusion}

%\vspace{-1.0mm}

We have presented \pdht, a partitioned, distributed hash table implementation
that uses a novel application of Portals networking interface constructs
initially intended for supporting MPI messaging.  By building \pdht on top of
Portals, we are able to take advantage of offload NICs that support the
Portals layer with a hardware processing engine.  \pdht is an example of a
family of partitioned, global logical address space (PGLAS) parallel
programming models and we anticipate that efficient implementations of such
models will be an enabling technology for sparse methods, which are of
increased interest within the scientific and data analytics communities.

In this initial design study, we have identified a first mapping of the \pdht
data structure to Portals primitives; however, there are a number of open areas
for future work.

Currently, we use a progress thread to process incoming insertion requests.  We
plan to investigate the use of the proposed Portals triggered ME append
operation~\cite{schneider:13} to allow \pdht to take full advantage of
asynchronous progress provided through the Portals layer.  However, in order
for this to work, we suspect changes to Portals may be necessary, as the ME
that is to be appended must also be updated with the user-supplied match bits
prior to being appended to the active list.

In Section~\ref{sec:results}, we identified list traversal as a significant
source of overhead.  A variety of solutions are possible, such as hashing
within the Portals implementation, and are also of potential benefit to MPI
implementations~\cite{flajslik:16}.

%%% Local Variables:
%%% mode: latex
%%% TeX-master: "paper"
%%% End:
