\section{Introduction}

Advances in chip design have led to a rise in the number of processor
cores within a single compute node. While this has added more capacity
for computation within high-performance computing (HPC) systems, it
comes at the expense of greater contention for shared resources within
the system.

Access to the interconnect network in HPC systems is an area where
this contention can cause a bottleneck, which may severely impact
communication throughput and overall system performance. To address
this problem, modern network interface hardware has been augmented
with significant computing power and additional circuitry to offload
many aspects of communication processing.

Offloading communication is typically performed by a network interface
stack, such as the Portals network programming interface from Sandia
or OpenFabrics' OFI/OFED. Among recent innovations in this
area, Atos/Bull has released their BXI interconnect fabric which
provides hardware support for the relatively new Portals 4.0
specification. Parallel applications that rely on middleware systems
constructed using the Portals interface will utilize new network
offload hardware automatically when run on systems that support it.

The Portals interface is provides a set of network access operations
that are intended to support efficient implementation of communication
middleware systems. Portals supports specific programming abstractions
useful to implement efficient systems for both message-passing and
one-sided systems.  Some interface/runtime systems focus on providing
high-performance message-passing semantics (e.g. OpenMPI, mvapich),
while others provide one-sided communication models to implement a
partitioned global address space (PGAS) (e.g. UPC, OpenSHMEM).

For message-passing, Portals provides a {\em matching interface}
framework for supporting a two-sided communication model based on
pairs of matched sends and receives. Portals also povides a {\em
  non-matching interface} that foregoes the complexity of a matched
two-sided communciations system in favor of a lightweight, efficient
system capable of one-sided operations used in PGAS systems.

While Portals was designed to be a foundation for efficient,
high-level communications systems, we have developed a distributed
key/value store that is build directly on top of Portals
primitives. The novel use of these primitive communication operations
provides a hardware accelerated parallel hash table system. In this
paper, we describe the implementation of a parallel distributed hash
table (PDHT) that is implemented using the Portals programming
interface. This work is based on the following insights: (1)
implementing data structures using Portals leads to an opportunity for
efficient implementation with respect to known hardware offload
engines and (2) that operations on a hash table can be readily mapped
to hardware accelerated operations within Portals that would be
difficult to express exclusively within a traditional message-passing
or PGAS system.

This work makes the following contributions: First, we describe the
design and implementation of a one-sided distributed hash table that
is amenable to network hardware offload in an HPC environment. Second,
we describe a novel application of the Portals matching interface
operations to realize a distributed data structure. Lastly, we provide
an experimental validation of our approach.


%%% Local Variables:
%%% mode: latex
%%% TeX-master: "paper"
%%% End:
