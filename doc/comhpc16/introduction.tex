\section{Introduction}

Advances in processor architecture have led to a rise in the number of
processor cores within a single compute node.  While this trend continues to
increase the computational capacity of high-performance computing (HPC)
systems, the amount of memory available per core has not increased at the same
rate.  As a result, there has been renewed interest in both Partitioned Global
Address Space (PGAS) parallel programming models as well as sparse and compact
data representations.

PGAS data models are supported by a number of industry standard parallel
programming frameworks, including Unified Parallel C~\cite{upc-13-spec},
OpenSHMEM~\cite{openshmem-1.3}, Fortran 2008 Coarrays~\cite{reid:08}, and the MPI
Remote Memory Access (RMA) interface~\cite{mpi-forum:15}.  Such models can be used
to aggregate the memory capacity of multiple nodes in an HPC system to provide
an efficient, distributed data store.  However, the data stored in the PGAS is
traditionally referenced using direct pointers or offsets with respect to the
memory where the data is stored.

This model has been effective for storing dense, regular data (e.g., dense
arrays~\cite{ga}); however, it presents significant challenges to
storing loosely-structured or sparse data.  For such data, calculating the
desired location in remote memory may not be possible.  Instead, one or more
lookups must be performed to identify the storage locations, leading to
significant runtime overheads.  When location resolution information is stored
remotely, the application incurs additional network latency overheads on each
access.  In some instances, the layout of the data may be fixed or slowly
changing, allowing location resolution structures to be replicated or cached
locally.  However, caching such structures consumes limited memory capacity and
can incur additional overheads to maintain cache coherence.

%Access to the interconnect network in HPC systems is an area where
%this contention can cause a bottleneck, which may severely impact
%communication throughput and overall system performance. To address
%this problem, modern network interface hardware has been augmented
%with significant computing power and additional circuitry to offload
%many aspects of communication processing.

The message matching model used by the Message Passing Interface (MPI)
establishes a layer of separation between a sender and the location where data
will be written at the receiver.  Instead of specifying a direct location where
data will be stored, the sender specifies an application-meaningful {\em tag}
that the receiver uses to associate the transmission with a data buffer.
However, while MPI provides a model for pushing data to a tagged location at a
remote node, it does not support retrieving data from a tagged location at a
remote node.

%Offloading communication is typically performed by a network interface
%stack, such as the Portals network programming interface from Sandia
%or OpenFabrics' OFI/OFED. Among recent innovations in this
%area, Atos/Bull has released their BXI interconnect fabric which
%provides hardware support for the relatively new Portals 4.0
%specification. Parallel applications that rely on middleware systems
%constructed using the Portals interface will utilize new network
%offload hardware automatically when run on systems that support it.

In this work, we build on networking structures designed to support MPI
communication in order to establish a partitioned, global logical address space
(PGLAS) that goes beyond traditional PGAS approaches by allowing applications
to address shared, distributed data using an application-meaningful logical
naming scheme rather than a direct memory address.  Our work utilizes the
Portals~\cite{portals4} network programming interface.  Portals provides a set
of communication primitives that are intended to efficiently support a variety
of HPC communication middleware systems, including MPI and PGAS models, while
also lending themselves to efficient hardware
implementation~\cite{brightwell:micro:06,bxi}.

%Portals supports specific programming abstractions
%useful to implement efficient systems for both message-passing and
%one-sided systems.  Some interface/runtime systems focus on providing
%high-performance message-passing semantics (e.g. OpenMPI, mvapich),
%while others provide one-sided communication models to implement a
%partitioned global address space (PGAS) (e.g. UPC, OpenSHMEM).
%
%For message-passing, Portals provides a {\em matching interface}
%framework for supporting a two-sided communication model based on
%pairs of matched sends and receives. Portals also povides a {\em
%  non-matching interface} that foregoes the complexity of a matched
%two-sided communciations system in favor of a lightweight, efficient
%system capable of one-sided operations used in PGAS systems.

Our approach demonstrates that existing Portals constructs designed to support
MPI messaging can be repurposed to support a PGLAS.  In particular, we have
developed a distributed
key/value store that is build directly on top of Portals
primitives. The novel use of these primitive communication operations
provides a hardware accelerated parallel hash table system. In this
paper, we describe the implementation of a parallel distributed hash
table (PDHT) that is implemented using the Portals programming
interface. This work is based on the following insights: (1)
implementing data structures using Portals leads to an opportunity for
efficient implementation with respect to known hardware offload
engines and (2) that operations on a hash table can be readily mapped
to hardware accelerated operations within Portals that would be
difficult to express exclusively within a traditional message-passing
or PGAS system.

This work makes the following contributions: First, we describe the
design and implementation of a one-sided distributed hash table that
is amenable to network hardware offload in an HPC environment. Second,
we describe a novel application of the Portals matching interface
operations to realize a distributed data structure. Lastly, we provide
an experimental validation of our approach.


%%% Local Variables:
%%% mode: latex
%%% TeX-master: "paper"
%%% End:
