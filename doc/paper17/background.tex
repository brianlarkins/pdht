\section{Background}

This work focuses on an efficient implementation of a runtime system that
provides a partitioned, global logical address space (PGLAS). Similar in spirit
to traditional PGAS data models, the PGLAS distinguishes itself by logically
addressing data elements with a {\em key}, which is meaningful to the
application. Keys are mapped to a unique location in the partitioned global
address space. The application may iterate over local keys or get locality
information for the logically addressed element in order to support
owner-compute work distribution. Rather than application-specific keys for distributed
data, conventional PGAS data models rely on direct remote addresses (e.g. OpenSHMEM
and UPC), an offset into a remote buffer (MPI), or an index into an array
(Fortran Coarrays). 

While PGLAS and PGAS models differ in addressing and layout, PGLAS shares a
one-sided communication model and asynchronous data accesss common to existing
PGAS approaches. Data is accessed using one-sided get, put, and atomic operations.
The PGLAS model is a member of the larger family of distributed key/value stores where
the organization and access of data follow a PGAS model.

% compare contrast PGAS/PGLAS
%   - PGAS requires global management of address space often with collectives (openshmem)
%   - PGLAS adds indirection layer between logical/physical global addresses
%   - reduces burden of global memory management and "enables more dynamic models"
%   - PGLAS can support dynamic put and get operations of key/value pairs in the
%     global address space

There are several important distinctions between the traditional global memory
addressing and management under the PGAS model versus the PGLAS approach. PGAS
models typically require global management of the addresses space. In the
OpenSHMEM model, for example, the globally accessible address space must be the
same on every process. This requires global coordination or the use of
collectives in order to effectively manage shared memory regions. In contrast,
the PGLAS model introduces an indirection layer between the logical and
physical global addresses. This avoids the overhead common to PGAS systems and
is used to relax the memory management semantics of such systems even further.
This indirection fosters more dynamic global memory usage, specifically with
the support for dynamic insertions and removals of key/value pairs in the
shared global address space. 

% efficiency
%   - must have efficient/scalable support for mapping keys -> addresses
%   - PGAS one-sided ops can do, but would require multiple communication
%     operations to resolve a logical address -> not efficient, high overhead

%   - MPI allows for applications to use tags to use tags associated with
%     the send and recieve operations
%   - MPI could be adapated to support a PGLAS model, using a unique tag for
%     each logical address     
%   - there are problems with this:
%     - MPI requires message ordering, but this is not required under the PGLAS model
%     - MPI requests for non-existent data elements are deferred on the unexpected
%       list, but not dropped. The PGLAS model requires that remote requests are
%       explicitly rejected in the case of a non-exsistent data element
%     - persistence: tags and tagged data elements exist only until a pairing 
%       can be made between sender and receiver, the PGLAS model requires that 
%       data elements (and the key/value association) continue to exist until
%       explicitly released.

Achieving high-performance with PGLAS is only possible with an efficient
mechanism for translating logical addresses (keys) to physical addresses within
a PGAS system. One-sided, asynchronous PGAS operations can be used to resolve
logical addresses, but it is difficult to do efficiently. Resolution may
require several communication operations to query a remote translation 
directory, resulting in high overheads~\cite{namashivayam:15}.

MPI provides application-specified tags to match up specific pairs of
send/receive operations. MPI tags could be used in place of more general PGLAS
keys, however there are limitations with this approach. First, MPI requires
ordered message processing, whereas both PGLAS and PGAS systems adopt
unordered-by-default communication models. Next, tags are intended to be used
as a mechanism to specify the matching of specific send/receive messages, not
to implement a persistent key/value store. Further, tagged receive operations are
consumed by incoming messages, whereas the PGLAS model permits the key/value
association to persist until explicitly released by the application. Lastly,
tagged MPI requests for data objects that do not exist will result in the
request being deferred to the unexpected message list, rather than a negative
acknowledgement. While this is reasonable for a two-sided communication model,
explicitly rejecting PGLAS requests for non-existent data objects is prudent.

\subsection{Distributed Sparse Data}

%XXX re-visit
We see the PGLAS model as a tool for dealing sparsity within HPC problem domains.  
Sparse data structures are commonly used in a number of important HPC 
applications, which can create challenges in obtaining high performance. Matrix 
and tensor representations frequently have to contend with sparse data, 
relying on specialized formats such as compressed sparse row (CSR) and variants.  
Other problems, such as grid/mesh solvers or N-body simulation are naturally 
expressed with spatial decomposition data structures such as quadtrees and 
octrees. 

In these instances, PGLAS provides a natural mapping between an application's 
inherent indexing system and the corresponding distributed data. Global 
distributed key/value stores make it straightforward to deal with sparse data 
and are helpful in avoiding the complications and difficulty in parallelizing 
these classes of applications.

\subsection{Related Work} 

Logically addressing data across memories in a distributed
system has been explored in a number of other models.  Distributed,
shared key/value stores are an intrinsic part of the Piccolo
programming model~\cite{power:10}. The Linda programming
model~\cite{ahuja:86} is built around tuple spaces that allow data
to be inserted, read, and removed from global memory. Linda data items
are identified through a user-supplied tuple. Such models are of
renewed interest in the context of resilience~\cite{wilke:14}.

Distributed object models, such as Orca~\cite{bal:92}, reference
shared data through graphs rather than direct pointers.  In the Open
Community Runtime (OCR)~\cite{OCR}, data blocks (DBs) are identified
by a globally unique ID, which is assigned by the OCR runtime system
and used to portably reference a DB.  In the MPI-3 RMA 
model~\cite{mpi-forum:15}, windows allowing remote
access to the private memory of a process can be created.  However,
window creation is collective and each process must maintain a table of
all window handles.  Finally, migratable object
models, such as Charm++~\cite{kale:93}, support a variety of methods
for representing and querying object collections.

Hash tables are also commonly used to store sparse or loosely structured
data~\cite{memcached04,chord01,docan:12} and their implementation has been
studied in the context of MPI and popular PGAS
models~\cite{zht13,fompi13,cmpi10,maynard:12,memcached12,mht15} as well as in
the context of RDMA offload~\cite{memcached12,mitchell:13,kalia:14}.  Others
have studied the applicability of programmable NICs to efficiently implement
key-value stores~\cite{li:17}.  In addition, PGAS-like hierarchical approaches
to managing sparse spatial decomposition have been proposed~\cite{larkins:08}
along with runtime methods to manage this data and exploit spatial
locality~\cite{larkins:12}.


%%% Local Variables:
%%% mode: latex
%%% TeX-master: "paper"
%%% End:
