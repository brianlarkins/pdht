\section{Conclusion}

This paper has presented the partitioned global address space (PGLAS)
programming model and a functioning implementation of the model in
PDHT. The PDHT system makes novel use of the Portals networking interface
by re-purposing the message-matching system to operate as a key mechanism
for a global, distributed key/value store. We envision that a number of
applications that rely on sparse methods can be readily adapted to the PGLAS
parallel programming model and have demonstrated that such systems can 
result in good performance.

This work has led to several insights into adapting Portals data abstractions
onto higher-level programming models. In particular, the separation of the
message matching mechanics from the requirement for ordered delivery is
critical to good performance with PDHT. We see future opportunities for
adapting the Portals triggered operations with separate semantics for capturing
the trigger parameters at setup time versus invocation time evaluation. Lastly, 
this work has motivated the need for permitting applications to search
the priority list for matching.

\section*{Acknowledgement}

This work used the Extreme Science and Engineering Discovery Environment
(XSEDE), which is supported by National Science Foundation grant number
ACI-1053575.

We would also like to acknowledge Ryan Grant and Todd Kordenbrock at Sandia
National Laboratories for their help with Portals-related issues and the
generous use of Sandia test bed facilities as well as Evangelos Georganas at
Intel for his assistance with the Meraculous benchmark.
