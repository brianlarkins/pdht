\section{Conclusion}

This paper has presented the partitioned global address space (PGLAS)
programming model and its first realization through the Parallel Distributed
Hash Table (PDHT).  PDHT system makes novel use of the Portals networking
interface by re-purposing its message-matching system to operate as an
acceleration mechanism for a global, distributed key/value store.

This work has led to several insights into adapting Portals data abstractions
onto higher-level programming models. In particular, the separation of the
message matching mechanics from the requirement for ordered delivery is
critical to good performance with PDHT. We see future opportunities for
adapting the Portals triggered operations to manage PDHT structures. Lastly,
this work has motivated several extensions to the Portals model, including
functionality allowing applications to search the priority list when responding
to local matching queries.

We envision that a number of applications that rely on sparse methods and
irregularly structured data can be readily adapted to the PGLAS parallel
programming model.  We identified two representative applications from the
computational chemistry and genomics domains and applied PDHT as a system for
managing the sparse spatial and loosely structured data arising in these
domains.  Experimental results are promising and indicate that Portals can
provide efficient and productive support for such data structures.

\begin{acks}
This work used the Extreme Science and Engineering Discovery Environment
(XSEDE), which is supported by National Science Foundation grant number
ACI-1053575.

We would also like to acknowledge Ryan Grant and Todd Kordenbrock at Sandia
National Laboratories for their help with Portals-related issues and the
generous use of Sandia test bed facilities as well as Evangelos Georganas at
Intel for his assistance with the Meraculous benchmark.
\end{acks}
